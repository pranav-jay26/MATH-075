\documentclass[12pt]{article}

\usepackage[utf8]{inputenc}
\usepackage{latexsym,amsfonts,amssymb,amsthm,amsmath,graphicx}

\setlength{\parindent}{0in}
\setlength{\oddsidemargin}{0in}
\setlength{\textwidth}{6.5in}
\setlength{\textheight}{8.8in}
\setlength{\topmargin}{0in}
\setlength{\headheight}{18pt}



\title{Math 075 Homework 4}
\author{Pranav Jayakumar}

\begin{document}

\maketitle

\subsection*{Exercise 2.3.7}
\vspace{0.25in}
\subsubsection*{Problem}
\vspace{0.25in}
\subsubsection*{Part (a)}
If $A^2$ can be formed, what can be said about the size of $A$?\\ 
\subsubsection*{Solution (a)}
If $A^2$ can be formed, this implies that a matrix must be square, since the number of rows and columns match.\\ 
\vspace{0.25in}
\subsubsection*{Part (b)}
If $AB$ and $BA$ can both be formed, what does this say about the sizes of bot $A$ and $B$?\\ 
\subsubsection*{Solution (b)}
When two matrices are multiplied together, the order in which they are multiplied affects the result, due to the fact that the dimensions must be matched. When $AB$ is equal to $BA$, it means that $A$ and $B$ are the same size.
\vspace{0.25in}
\subsubsection*{Part (c)}
If $ABC$ can be formed, $A = 3\times 3$, and $C = 5 \times 5$, what size is $B$?\\ 
\subsubsection*{Solution (c)}
For two matrices $A$ and $B$ to be multiplied together, the number of columns in $A$ must be equal to the number of rows in $B$. For $ABC$ to be a valid matrix multiplication, the number of columns in $A$ must be equal to the number of rows in $B$, and the number of columns in $B$ must be equal to the number of rows in $C$. Therefore, $B$ has dimensions $3 \times 5$.\\ 
\vspace{2in}
\subsection*{Exercise 2.3.9}
\vspace{0.25in}
\subsubsection*{Problem}
Write P = 
$\begin{bmatrix}
  1 & 0 & 0\\ 
  0 & 0 & 1\\ 
  0 & 1 & 0 
\end{bmatrix}$, 
and let $A$ be $3 \times n$ and $B$ be $m \times 3$. 
\vspace{0.25in}
\subsubsection*{Part (a)}
Describe $PA$ in terms of the rows of $A$. 
\vspace{0.25in}
\subsubsection*{Solution (a)}
For all $PA$ where $P$ is size $3 \times 3$, and $A$ is size $3 \times n$, $PA$ would have dimensions $3 \times n$.\\ 
\[
  P = 
  \begin{bmatrix}
    1 & 0 & 0\\ 
    0 & 0 & 1\\ 
    0 & 1 & 0 
  \end{bmatrix}
  ,\hspace{0.1in}
  A = 
  \begin{bmatrix}
    a_{11} & a_{12} & a_{13} & \cdots & a_{1n}\\ 
    a_{21} & a_{22} & a_{23} & \cdots & a_{2n}\\ 
    a_{31} & a_{32} & a_{33} & \cdots & a_{3n} 
  \end{bmatrix}
\]
\[
  PA = 
  \begin{bmatrix}
    pa_{11} & pa_{12} & pa_{13} & \cdots & pa_{1n} 

  \end{bmatrix}, \hspace{0.1in}
  pa_{ij} = \displaystyle\sum_{k = 1}^{n} p_{ik} \cdot a_{kj}
\]
\vspace{0.25in}
\subsubsection*{Part (b)}
\[
  P = 
  \begin{bmatrix}
    1 & 0 & 0\\ 
    0 & 0 & 1\\ 
    0 & 1 & 0 
  \end{bmatrix}, \hspace{0.1in}
  B = 
  \begin{bmatrix}
    b_{11} & b_{12} & b_{13}\\ 
    b_{21} & b_{22} & b_{23}\\ 
    b_{31} & b_{32} & b_{33} \\ 
    \vdots & \vdots & \vdots\\ 
    b_{m1} & b_{m2} & b_{m3} 
  \end{bmatrix}
\]
\[
  PB = 
  \begin{bmatrix}
    pb_1 & pb_2 & pb_3 
  \end{bmatrix}, \hspace{0.1in}
  pb_i = \displaystyle\sum_{k = 1}^{3}\displaystyle\sum_{j = 1}^{m} p_{ik} \cdot b_{kj}
\]
\vspace{2in}
\subsection*{Exercise 2.4.2}
\vspace{0.25in}
\subsubsection*{Problem}
Find the inverse of each of the following matrices. 
\vspace{0.25in}
\subsubsection*{Part (c)}
\[
  \begin{bmatrix}
    1 & 0 & -1\\ 
    3 & 2 & 0\\ 
    -1 & -1 & 0 
  \end{bmatrix}
\]
\subsubsection*{Solution (c)}
\[
  A \cdot A^{-1} = I_{n}
\]
\[
  A = 
  \begin{bmatrix}
    1 & 0 & -1\\ 
    3 & 2 & 0\\ 
    -1 & -1 & 0 
  \end{bmatrix}, \hspace{0.1in}
  A^{-1} = RREF\left(
    \begin{bmatrix}
      1 & 0 & -1 & \bigm| & 1 & 0 & 0\\ 
      3 & 2 & 0 & \bigm| & 0 & 1 & 0\\ 
      -1 & -1 & 0 & \bigm| & 0 & 0 & 1 
    \end{bmatrix}\right)
\]
\[
  \begin{bmatrix}
    1 & 0 & -1 & \bigm| & 1 & 0 & 0\\ 
    3 & 2 & 0 & \bigm| & 0 & 1 & 0\\ 
    -1 & -1 & 0 & \bigm| & 0 & 0 & 1 
  \end{bmatrix}
  \overset{R_3 \leftrightarrow R_2}{\longrightarrow}
  \begin{bmatrix}
    1 & 0 & -1 & \bigm| & 1 & 0 & 0\\ 
    -1 & -1 & 0 & \bigm| & 0 & 0 & 1\\ 
    3 & 2 & 0 & \bigm| & 0 & 1 & 0 
  \end{bmatrix}
\]
\[
  \overset{R_2 \rightarrow R_2 + R_1}{\longrightarrow}
  \begin{bmatrix}
    1 & 0 & -1 & \bigm| & 1 & 0 & 0\\ 
    0 & -1 & -1 & \bigm| & 1 & 0 & 1\\ 
    3 & 2 & 0 & \bigm| & 0 & 1 & 0 
  \end{bmatrix}
  \overset{R_3 \rightarrow R_3 - 3R_1}{\longrightarrow}
  \begin{bmatrix}
    1 & 0 & -1 & \bigm| & 1 & 0 & 0\\ 
    0 & -1 & -1 & \bigm| & 1 & 0 & 1\\ 
    0 & 2 & 3 & \bigm| & -3 & 1 & 0 
  \end{bmatrix}
\]
\[
  \overset{R_3 \rightarrow R_3 + 2R_2}{\longrightarrow}
  \begin{bmatrix}
    1 & 0 & -1 & \bigm| & 1 & 0 & 0\\ 
    0 & -1 & -1 & \bigm| & 1 & 0 & 1\\ 
    0 & 0 & 1 & \bigm| & -1 & 1 & 2 
  \end{bmatrix}
  \overset{R_2 \rightarrow R_2 + R_3}{\longrightarrow}
  \begin{bmatrix}
    1 & 0 & -1 & \bigm| & 1 & 0 & 0\\ 
    0 & -1 & 0 & \bigm| & 0 & 1 & 3\\ 
    0 & 0 & 1 & \bigm| & -1 & 1 & 2 
  \end{bmatrix}
\]
\[
  \overset{R_2 \rightarrow -R_2}{\longrightarrow}
  \begin{bmatrix}
    1 & 0 & -1 & \bigm| & 1 & 0 & 0\\ 
    0 & 1 & 0 & \bigm| & 0 & -1 & -3\\ 
    0 & 0 & 1 & \bigm| & -1 & 1 & 2 
  \end{bmatrix}
  \overset{R_1 \rightarrow R_1 + R_3}{\longrightarrow}
  \begin{bmatrix}
    1 & 0 & 0 & \bigm| & 0 & 1 & 2\\ 
    0 & 1 & 0 & \bigm| & 0 & -1 & -3\\ 
    0 & 0 & 1 & \bigm| & -1 & 1 & 2 
  \end{bmatrix}
\]
\vspace{0.25in}
\[
  A^{-1} = 
  \begin{bmatrix}
    0 & 1 & 2\\ 
    0 & -1 & -3\\ 
    -1 & 1 & 2 
  \end{bmatrix}
\]
\vspace{2in}
\subsection*{Exercise 2.4.4}
\vspace{0.25in}
\subsubsection*{Problem}
We are given the matrix
$A^{-1} =  
\begin{bmatrix}
  1 & -1 & 3\\ 
  2 & 0 & 5\\ 
  -1 & 1 & 0 
\end{bmatrix}
$. 
\vspace{0.25in}
\subsubsection*{Part (a)}
Solve the system of equations
$A\textbf{x} = 
\begin{bmatrix}
  1\\ 
  -1\\ 
  3 
\end{bmatrix}.
$ 
\vspace{0.25in}
\subsubsection*{Solution (a)}
\[
  A\textbf{x} = \textbf{b}
\]
\[
  A^{-1} \cdot A\textbf{x} = I_{n}\textbf{b}
\]
\[
  \begin{bmatrix}
    1 & -1 & 3\\ 
    2 & 0 & 5\\ 
    -1 & 1 & 0 
  \end{bmatrix}
  \begin{bmatrix}
    1\\ 
    -1\\ 
    3 
  \end{bmatrix}
  = 
  I_{n}\textbf{b}
\]
\[
  1\begin{bmatrix}
    1\\ 
    2\\ 
    -1 
  \end{bmatrix}
  - 1\begin{bmatrix}
    -1\\ 
    0\\ 
    1 
  \end{bmatrix}
  + 3\begin{bmatrix}
    3\\ 
    5\\ 
    0 
  \end{bmatrix}
  = \textbf{x}
\]
\vspace{0.25in}
\[
  \textbf{x} = 
  \begin{bmatrix}
    11\\ 
    17\\ 
    -2 
  \end{bmatrix}
\]
\vspace{0.25in}
\subsubsection*{Part (c)}
Find a matrix $C$ such that $CA = 
\begin{bmatrix}
  1 & 2 & -1\\ 
  3 & 1 & 1 
\end{bmatrix}$. 
\vspace{0.25in}
\subsubsection*{Solution (c)}
\[
  CAA^{-1} = CI 
\]
\[
  \begin{bmatrix}
    1 & 2 & -1\\ 
    3 & 1 & 1 
  \end{bmatrix}
  \begin{bmatrix}
    1 & -1 & 3\\ 
    2 & 0 & 5\\ 
    -1 & 1 & 0 
  \end{bmatrix}
  = C 
\]
\[
  C =
  \begin{bmatrix}
    1 - 1 + 3 & 4 + 0 + 10 & 1 - 1 + 0\\ 
    3 - 3 + 9 & 2 + 0 + 5 & -1 + 1 + 0 
  \end{bmatrix}
\]
\vspace{0.25in}
\[
  C = 
  \begin{bmatrix}
    3 & 14 & 0\\ 
    9 & 7 & 0 
  \end{bmatrix}
\]
\vspace{2in}
\subsection*{Exercise 2.4.5}
\vspace{0.25in}
\subsubsection*{Problem}
Find $A$. 
\vspace{0.25in}
\subsubsection*{Part (e)}
\[
  \left(
    A 
    \begin{bmatrix}
      1 & -1\\ 
      0 & 1 
    \end{bmatrix}
  \right)^{-1}
  = 
  \begin{bmatrix}
    2 & 3\\ 
    1 & 1 
  \end{bmatrix}
\]
\vspace{0.25in}
\subsubsection*{Solution (e)}
\[
  A 
  \begin{bmatrix}
    1 & -1\\ 
    0 & 1 
  \end{bmatrix}
  = 
  \begin{bmatrix}
    2 & 3\\ 
    1 & 1 
  \end{bmatrix}^{-1}
\]
\[
  A 
  \begin{bmatrix}
    1 & -1\\ 
    0 & 1 
  \end{bmatrix}
  = \frac{1}{2 - 3} 
  \begin{bmatrix}
    1 & -3\\ 
    -1 & 2 
  \end{bmatrix}
\]
\[
  A 
  \begin{bmatrix}
    1 & -1\\ 
    0 & 1 
  \end{bmatrix}
  = -1
  \begin{bmatrix}
    1 & -3\\ 
    -1 & 2 
  \end{bmatrix}
\]
\[
  A = 
  \begin{bmatrix}
    1 & -1\\ 
    0 & 1 
  \end{bmatrix}^{-1}
  \cdot -1
  \begin{bmatrix}
    1 & -3\\ 
    -1 & 2 
  \end{bmatrix}
\]
\[
  A = 
  \begin{bmatrix}
    1 & 1\\ 
    0 & 1 
  \end{bmatrix}
  \cdot 
  \begin{bmatrix}
    -1 & 3\\ 
    1 & -2 
  \end{bmatrix}
\]
\[
  A = 
  \begin{bmatrix}
    -1 + 3 & 1 - 2\\ 
    0 + 0 & 1 - 2 
  \end{bmatrix}
\]
\[
  A = 
  \begin{bmatrix}
    2 & -1\\ 
    0 & -1 
  \end{bmatrix}
\]
\vspace{0.25in}
\subsubsection*{Part (g)}
\[
  \left(A^T - 2I\right)^{-1} = 2\begin{bmatrix}
    1 & 1\\ 
    2 & 3 
  \end{bmatrix}
\]
\vspace{0.25in}
\subsubsection*{Solution (g)}
\[
  A^T - 2I = 2\begin{bmatrix}
    1 & 1\\ 
    2 & 3 
  \end{bmatrix}^{-1}
\]
\[
  A^T - 2I = 2 \cdot \frac{1}{3 - 2}\begin{bmatrix}
    3 & -1\\ 
    -2 & 1 
  \end{bmatrix}
\]
\[
  A^T = 2\begin{bmatrix}
    3 & -1\\ 
    -2 & 1 
  \end{bmatrix} + 2I 
\]
\[
  A^T = 2\begin{bmatrix}
    5 & -1\\ 
    -2 & 3 
  \end{bmatrix}
\]
\[
  A = 2\begin{bmatrix}
    5 & -1\\ 
    -2 & 3 
  \end{bmatrix}^T 
\]
\vspace{0.25in}
\[
  A = 2\begin{bmatrix}
    5 & -2\\ 
    -1 & 3 
  \end{bmatrix}
\]
\vspace{0.5in}
\end{document}
