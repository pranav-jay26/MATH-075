\documentclass[12pt]{article}

\usepackage[utf8]{inputenc}
\usepackage{latexsym,amsfonts,amssymb,amsthm,amsmath,graphicx}

\setlength{\parindent}{0in}
\setlength{\oddsidemargin}{0in}
\setlength{\textwidth}{6.5in}
\setlength{\textheight}{8.8in}
\setlength{\topmargin}{0in}
\setlength{\headheight}{18pt}



\title{Math 075 Homework 3}
\author{Pranav Jayakumar}

\begin{document}

\maketitle

\subsection*{Exercise 2.1.14}
\vspace{0.25in}
\subsubsection*{Problem}
In each case determine all $s$ and $t$ such that the given matrix is symmetric
\subsubsection*{Part (c)}
\[
\begin{bmatrix}
    s & 2s & st\\
    t & -1 & s\\
    t & s^2 & s
\end{bmatrix}
\]
\subsubsection*{Solution (c)}
\begin{equation*}
  \begin{cases}
    t = 2s\\ 
    t = st\\ 
    s^2 = s
  \end{cases}
\end{equation*}
\begin{equation*}
  \begin{cases}
    s = 1\\ 
    t = 2 
  \end{cases}
\end{equation*}
\vspace{0.25in}
The given matrix is symmetric when $s = 1$ and $t = 2$.
\vspace{2in}
\subsection*{Exercise 2.2.15}
\vspace{0.25in}
\subsubsection*{Problem}
In each case find the matrix A. 
\subsubsection*{Part (a)}
\[
  \begin{equation*}
    \left(A + 3\begin{bmatrix}
        1 & -1 & 0\\ 
        1 & 2 & 4 
    \end{bmatrix}^T
    = \begin{bmatrix}
      2 & 1\\ 
      0 & 5\\ 
      3 & 8 
    \end{bmatrix}
    \right)
  \end{equation*}
\]
\subsubsection*{Solution (a)}
\begin{equation*}
  A + 3\begin{bmatrix}
    1 & 1\\ 
    -1 & 2\\ 
    0 & 4 
  \end{bmatrix}
  = \begin{bmatrix}
    2 & 1\\ 
    0 & 5\\ 
    3 & 8 
  \end{bmatrix}
\end{equation*}
\begin{equation*}
  A + \begin{bmatrix}
    3 & 3\\ 
    -3 & 6\\ 
    0 & 12 
  \end{bmatrix}
  = \begin{bmatrix}
    2 & 1\\ 
    0 & 5\\ 
    3 & 8 
  \end{bmatrix}
\end{equation*}
\begin{equation*}
  A = \begin{bmatrix}
    -1 & -2\\ 
    3 & -1\\ 
    3 & -4
  \end{bmatrix}
\end{equation*}
\vspace{2in}
\subsection*{Exercise 2.2.1}
\vspace{0.25in}
\subsubsection*{Problem}
In each case find a system of equations that is equal to the vector equation. 
\vspace{0.25in}
\subsubsection*{Part (a)}
\begin{equation*}
  x_1\begin{bmatrix}
  2\\ 
  -3\\ 
  0 
\end{bmatrix}
+ x_2\begin{bmatrix}
  1\\ 
  1\\ 
  4 
\end{bmatrix}
+ x_3\begin{bmatrix}
  2\\ 
  0\\ 
  -1 
\end{bmatrix}
 = \begin{bmatrix}
   5\\ 
   6\\ 
   3 
 \end{bmatrix}
\end{equation*}
\subsubsection*{Solution (a)}
\begin{equation*}
  \begin{cases}
    2x_1 + x_2 + 2x_3 = 5\\ 
    -3x_1 + x_2 = 6\\ 
    4x_2 - x_3 = 3 
  \end{cases}
\end{equation*}
\[
  A = \begin{bmatrix}
  2 & 1 & 2 & 5\\ 
  -3 & 1 & 0 & 6\\ 
  0 & 4 & -1 & 3 
\end{bmatrix}
\]
\vspace{2in}
\subsection*{Exercise 2.2.5}
\vspace{0.25in}
\subsubsection*{Problem}
In each case, express every solution of the system as a sum of a specific soution plus a solution of the associated homogeneous system. 
\vspace{0.25in}
\subsubsection*{Part (c)}
\begin{equation*}
  \begin{cases}
    x_1 + x_2 - x_3 - 5x_5 = 2\\ 
    x_2 + x_3 - 4x_5 = -1\\ 
    x_2 + x_3 + x_4 - x_5 = -1\\ 
    -2x_1 - x_2 + 2x_4 = 3 
  \end{cases}
\end{equation*}
\vspace{0.25in}
\subsubsection*{Solution (c)}
\[
  \begin{bmatrix}
    1 & 1 & -1 & 0 & -5 & 2\\ 
    0 & 1 & 1 & 0 & -4 & -1\\ 
    0 & 1 & 1 & 1 & -1 & -1\\ 
    0 & 0 & -4 & 1 & 1 & 6 
  \end{bmatrix}
  \overset{RREF}{\longrightarrow}
  \begin{bmatrix}
    1 & 0 & 0 & 0 & 0 & 0\\ 
    0 & 1 & 0 & 0 & -\frac{9}{2} & \frac{1}{2}\\ 
    0 & 0 & 1 & 0 & \frac{1}{2} & -\frac{3}{2}\\ 
    0 & 0 & 0 & 1 & 3 & 0 
  \end{bmatrix}
\]
\begin{equation*}
  \begin{cases}
    x_1 + 0x_5 = 0\\ 
    x_2 - \frac{9}{2}x_5 = \frac{1}{2}\\ 
    x_3 + \frac{1}{2}x_5 = -\frac{3}{2}\\ 
    x_4 + 3x_5 = 0\\ 
    x_5 = free
  \end{cases}
\end{equation*}
\[
  \begin{bmatrix}
    1 & 1 & -1 & - & -5 & 0\\ 
    0 & 1 & 1 & 0 & -4 & 0\\ 
    0 & 1 & 1 & 1 & -1 & 0\\ 
    0 & 0 & -4 & 1 & 1 & 0 
  \end{bmatrix}
  \overset{RREF}{\longrightarrow}
  \begin{bmatrix}
    1 & 0 & 0 & 0 & 0 & 0\\ 
    0 & 1 & 0 & 0 & -\frac{9}{2} & 0\\ 
    0 & 0 & 1 & 0 & \frac{1}{2} & 0\\ 
    0 & 0 & 0 & 1 & 3 & 0 
  \end{bmatrix}
\]
\begin{equation*}
  \begin{cases}
    x_1 + 0x_5 = 0\\ 
    x_2 - \frac{9}{2}x_5 = 0\\ 
    x_3 + \frac{1}{2}x_5 = 0\\ 
    x_4 + 3x_5 = 0\\ 
    x_5 = free
  \end{cases}
\end{equation*}
\begin{equation*}
  \begin{bmatrix}
    x_1\\ 
    x_2\\ 
    x_3\\ 
    x_4\\ 
    x_5 
  \end{bmatrix} =
  \begin{bmatrix}
    0\\ 
    \frac{1}{2}\\ 
    -\frac{3}{2}\\ 
    0\\ 
    0 
  \end{bmatrix}
  + s 
  \begin{bmatrix}
    0\\ 
    \frac{9}{2}\\ 
    -\frac{1}{2}\\ 
    -3\\ 
    1 
  \end{bmatrix}
  + t 
  \begin{bmatrix}
    0\\ 
    \frac{9}{2}\\ 
    -\frac{1}{2}\\ 
    -3\\ 
    1 
  \end{bmatrix}
\end{equation*}
\vspace{2in}
\subsection{Exercise 2.2.12}
\vspace{0.25in}
\subsubsection*{Problem}
The projection $P : \mathbb{R}^3 \rightarrow \mathbb{R}^2$ is defined by $P\begin{bmatrix}
  x\\ 
  y\\ 
  z 
\end{bmatrix} 
= 
\begin{bmatrix}
  x\\ 
  y 
\end{bmatrix}
$ 
for all
$\begin{bmatrix}
  x\\ 
  y\\ 
  z 
\end{bmatrix}
$ 
in $\mathbb{R}^3$. Show that $P$ is induced by a matrix and find the matrix. 

\end{document}

